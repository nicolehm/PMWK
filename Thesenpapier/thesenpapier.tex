\documentclass{pmwk}
\usepackage{ngerman}
\usepackage[utf8]{inputenc}
\usepackage{color}
\usepackage{xcolor}
\usepackage[colorinlistoftodos]{todonotes}
\usepackage{amsmath}
\usepackage{amssymb}
\usepackage{amsthm}
\usepackage{listings}
\usepackage{hyperref}
\usepackage{breakurl}
\usepackage{algorithm2e}
\begin{document}

\Abgabeblatt{}{}{}{}




\section*{Glaucia}
TODO
\section*{Elena}
TODO
\section*{Dana}
Ein Staat hat besonderes Interesse daran die Wissenschaft zu schützen und zu fördern. Einerseits um politische Entscheidungen gegenüber der Gesellschaft zu begründen (Huber, 2008, S. 11), andererseits bringt eine florierende Wissenschaft einen wirtschaftlichen Fortschritt des Staates (Huber, 2008, S. 35). Für die Wissenschaft schafft der Staat im Gegenzug eine Rechtsordnung in der sie sich entfalten kann und finanziert sie. Grundlage dafür ist in Deutschland Paragraph 3 Artikel 5 Meinungsfreiheit des Grundgesetzes:\par
„Kunst und Wissenschaft, Forschung und Lehre sind frei. Die Freiheit der Lehre entbindet nicht von der Treue zur Verfassung.“ [GG Art 5 § 3]\par
Es gibt verschiedene herrschende Ansichten über die Auslegung des Paragraphen. Gängige Meinung der Rechtsprechung und –lehre ist aber, dass Forschung und Lehre im Begriff Wissenschaft zusammengefasst werden (Wagner, 2000, S. 27). Laut einem Urteil des Bundesverfassungsgerichtes ist durch den Paragraphen jede Tätigkeit geschützt, die „…nach Inhalt und Form als ernsthafter planmäßiger Versuch zur Ermittlung der Wahrheit anzusehen ist“ [BVerfGE 35,79 vom 29.05.1973]. Dabei ist vor allem wichtig, dass das Ergebnis keine Relevanz darauf hat, ob die vorhergehende Wissenschaft schützenswert ist. \textcolor{red}{TODO Schranken}

\section*{Nico}
TODO
\section*{Brian}
%TODO proofread and spellcheck
Wie finanzieren die Hochschulen ihre Forschung und wie wird dadurch deren Forschungsfreiheit beeinflusst? Zum einen gibt es die Erstmittelfinanzierung. Diese ist aber für unser Referat Relativ uninterressant. Die Erstmittel werden vom zuständigen Ministerium bereitgestellt. Die Drittmittel hingegen sind Ressourcen, welche von aussen gesponsort werden. Dies geschiet entweder privat, kommt aus der Wirtschaft oder wird von Forschungsfördernden Vereinen bereitgestellt. Neben den Vorteilen sind hier besonders die Nachteile interessant. Die Aussenstehenden stellen oft bestimmte Forderungen die erfüllt werden müssen, damit sie Gelder zahlen. Dies sollte man kritisch hinterfragen. Ist Forschung wirklich noch frei, wenn sie durch Aussenstehende eingeschränkt wird? Das DFG (Deutsche Forschungsgemeinschaft) bewertet zudem noch die Forscher anhand der Menge der Drittmittel die sie werben. Hierbei läuft die Gefahr, dass Forscher nicht mehr Forschen sondern, in diesem Falle bewusst übertrieben Formuliert, Geldeintreiber sind. Desweiteren hat sich das Referat mit den Unterschieden der Forschung zwischen Unternehmen und Hochschulen beschäftigt. Wir sind dabei zu dem Schluss gekommen, dass Unternehmen sehr gewinnorientiert Forschen. Die Forschung findet zudem meist hinter geschlossenen Türen statt, so dass die Algemeinheit keinen direkten Zugriff darauf hat. Ausserdem kann es vorkommen, das unerwünschte Forschungsergebnisse einfach vernichtet werden.

\section*{Kristina}
TODO

\end{document}