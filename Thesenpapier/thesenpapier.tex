\documentclass{pmwk}
\usepackage{ngerman}
\usepackage[utf8]{inputenc}
\usepackage{color}
\usepackage{xcolor}
\usepackage[colorinlistoftodos]{todonotes}
\usepackage{amsmath}
\usepackage{amssymb}
\usepackage{amsthm}
\usepackage{listings}
\usepackage{hyperref}
\usepackage{breakurl}
\usepackage{algorithm2e}

\begin{document}
\nocite{01,02,03,04,05,06,DeuFunk,forschfoerd,fiff,gvw}


\Abgabeblatt{}{}{}{}




\section*{Glaucia}
TODO

\section*{Elena}
Obwohl in Deutschland Forschungsfreiheit gilt, wird nicht alleine durch die Wissenschaft entschieden, was für Forschungen und Entwicklungen betrieben werden. Gerade über die Finanzierung kann die Gesellschaft und die Politik viel Einfluss auf die Wissenschaft nehmen. Aber auch andersherum nimmt die Wissenschaft Einfluss auf die Politik und die Gesellschaft. \par
Die Politik beeinflusst die Wissenschaft durch Gesetze, dabei werden entweder Bereiche verboten oder durch Gesetze Märkte für neue Techniken geschaffen, um die Erforschung oder Entwicklung voranzutreiben. Auch durch Fördermittel, die für bestimmte Bereiche ausgeschrieben werden, kann die Politik Einfluss darauf nehmen, was geforscht wird. Auf der anderen Seite begründet die Politik viele Gesetze mithilfe von wissenschaftlichen Veröffentlichungen und ist somit auf wissenschaftliche Ergebnisse angewiesen. \par
Die Wissenschaft beeinflusst zum Beispiel durch die Medizin und neue Technologien ganz entscheidend unser heutiges Leben. Aber auch die Gesellschaft hat starken Einfluss auf die Wissenschaft. Bei großer Nachfrage wird viel Geld für die Forschung und Entwicklung in den entsprechenden Bereichen bereitgestellt. Wenn mehrere Unternehmen in einem Bereich Forschen, beschleunigt dies durch Konkurrenz auch die Geschwindigkeit in der Ergebnisse erwartet werden. 

\section*{Dana}
Ein Staat hat besonderes Interesse daran die Wissenschaft zu schützen und zu fördern. Einerseits um politische Entscheidungen gegenüber der Gesellschaft zu begründen \cite[11]{Huber}, andererseits bringt eine florierende Wissenschaft einen wirtschaftlichen Fortschritt des Staates \cite[35]{Huber}. Für die Wissenschaft schafft der Staat im Gegenzug eine Rechtsordnung in der sie sich entfalten kann und finanziert sie. Grundlage dafür ist in Deutschland Paragraph 3 Artikel 5 Meinungsfreiheit des Grundgesetzes:\par
„Kunst und Wissenschaft, Forschung und Lehre sind frei. Die Freiheit der Lehre entbindet nicht von der Treue zur Verfassung.“ \cite[Art 5 § 3]{gg}\par
Es gibt verschiedene herrschende Ansichten über die Auslegung des Paragraphen. Gängige Meinung der Rechtsprechung und –lehre ist aber, dass Forschung und Lehre im Begriff Wissenschaft zusammengefasst werden \cite[27]{RechFrei}. Laut einem Urteil des Bundesverfassungsgerichtes ist durch den Paragraphen jede Tätigkeit geschützt, die „…nach Inhalt und Form als ernsthafter planmäßiger Versuch zur Ermittlung der Wahrheit anzusehen ist“ [BVerfGE 35,79 vom 29.05.1973]. Dabei ist vor allem wichtig, dass das Ergebnis keine Relevanz darauf hat, ob die vorhergehende Wissenschaft schützenswert ist \cite[13]{WisArb}. Die Wissenschaftsfreiheit findet seine Schranken lediglich bei der Überschreitung anderer Grundrechte\cite[61]{Huber}.

\section*{Nico}
Auch außerhalb von Gesetzlichen Rahmenbedingungen gibt es Bestrebungen die Forschungsfreiheit zu begrenzen und die Forschung moralischen Regularien zu unterwerfen.\par

Das bekannteste Mittel sind hier die Ethikkommissionen an Hochschulen und privaten Forschungseinrichtungen. Eine Ethikkommission bei einem neuen Forschungsvorhaben hinzuzuziehen ist zwar gesetzlich verpflichtend für klinische Studien und Tierversuchen, jedoch wird deren Kompetenz oftmals auch auf weitere Resorts ausgeweitet wie zum Beispiel in der Gentechnik oder wenn mit personenbezogenen Daten geforscht wird.\par

Ein weiteres selbstverpflichtendes Mittel zur Beschränkung der Forschungsfreiheit ist eine Zivilklausel. Diese schreibt vor, dass die betreffende wissenschaftliche Einrichtung ausschließlich für zivilie Zwecke forscht. Beginnend mit der Zivilklausel der Universität Bremen von 1986, haben immer mehr Einrichtungen eine Zivilklausel eingeführt. Eine solche Zivilklausel ist nicht unumstritten, vor allem in Bezug auf die gesetzlich garantierte Freiheit von Forschung als auch bezogen auf die Dual Use Frage (viele Forschungsergebnisse können sowohl zivil als auch militärisch eingesetzt werden) \cite{JKrause}. 

\section*{Brian}
Wie finanzieren die Hochschulen ihre Forschung und wie wird dadurch deren Forschungsfreiheit beeinflusst? Zum einen gibt es die Erstmittelfinanzierung, welche vom zuständigen Ministerium bereitgestellt werden. Die Drittmittel hingegen sind Ressourcen, welche von außen gesponsert werden. Dies geschieht entweder privat, kommt aus der Wirtschaft oder wird von forschungsfördernden Vereinen bereitgestellt. Neben den Vorteilen sind hier besonders die Nachteile interessant. Die Aussenstehenden stellen oft bestimmte Forderungen die erfüllt werden müssen, damit sie Gelder zahlen. Dies sollte man kritisch hinterfragen. Ist Forschung wirklich noch frei, wenn sie durch Aussenstehende eingeschränkt wird? Das \textit{Deutsche Forschungsgemeinschaft} (DFG) bewertet zudem noch die Forscher anhand der Menge der Drittmittel die sie werben. Hierbei läuft die Gefahr, dass Forscher nicht mehr Forschen sondern, in diesem Falle bewusst übertrieben formuliert, Geldeintreiber sind.  Die Forschung in Unternehmen findet zudem meist hinter geschlossenen Türen statt, so dass die Allgemeinheit keinen direkten Zugriff darauf hat. Außerdem kann es vorkommen, das unerwünschte Forschungsergebnisse einfach vernichtet werden.

\section*{Kristina}
Der Abwurf der Atombomben auf Hiroshima und Nagasaki im Jahr 1945 forderte circa 200000 Menschen und befähigte die Menschheit zudem sich selber zu vernichten. Diese Ausmaße der aus wissenschaftlichen Erkenntnissen entspringenden technischen Macht bewirkte in der Gesellschaft erstmals das Bedürfnis über die Wissenschaft, ihre Methoden und Ziele zu reflektieren. Die in den USA gegründete \textit{Society for Social Responsibility in Science (SSRS)} war die erste Vereinigung, deren Mitglieder sich für den verantwortungsvollen Umgang mit Wissenschaft einsetzten und den Dialog zwischen Vertretern der Wissenschaft und Gesellschaft forderten. Die SSRS war Vorbild für eine Zahl weltweit entstehender Gesellschaften. Neben den Vereinigungen, die für alle Wissensgebiete offen sind wie zum Beispiel die \textit{Gesellschaft für Verantwortung in der Wissenschaft (GVW)} enstanden auch Gesellschaften,  die sich für ein Spezialgebiet einsetzten. Das \textit{Forum für InformatikerInnen für Frieden und gesellschaftliche Verantwortung e.V. (FIfF)} verfolgt unter anderem das Ziel die Öffentlichkeit über Entwicklungen aus der Informatik und Informationstechnik aufzuklären.

\bibliography{sources.bib}
\bibliographystyle{alpha}
\end{document}
