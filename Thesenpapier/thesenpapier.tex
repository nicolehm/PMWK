\documentclass{pmwk}
\usepackage{ngerman}
\usepackage[utf8]{inputenc}
\usepackage{color}
\usepackage{xcolor}
\usepackage[colorinlistoftodos]{todonotes}
\usepackage{amsmath}
\usepackage{amssymb}
\usepackage{amsthm}
\usepackage{listings}
\usepackage{hyperref}
\usepackage{breakurl}
\usepackage{algorithm2e}
\begin{document}

\Abgabeblatt{}{}{}{}




\section*{Glaucia}
TODO
\section*{Elena}
TODO
\section*{Dana}
Ein Staat hat besonderes Interesse daran die Wissenschaft zu schützen und zu fördern. Einerseits um politische Entscheidungen gegenüber der Gesellschaft zu begründen (Huber, 2008, S. 11), andererseits bringt eine florierende Wissenschaft einen wirtschaftlichen Fortschritt des Staates (Huber, 2008, S. 35). Für die Wissenschaft schafft der Staat im Gegenzug eine Rechtsordnung in der sie sich entfalten kann und finanziert sie. Grundlage dafür ist in Deutschland Paragraph 3 Artikel 5 Meinungsfreiheit des Grundgesetzes:\par
„Kunst und Wissenschaft, Forschung und Lehre sind frei. Die Freiheit der Lehre entbindet nicht von der Treue zur Verfassung.“ [GG Art 5 § 3]\par
Es gibt verschiedene herrschende Ansichten über die Auslegung des Paragraphen. Gängige Meinung der Rechtsprechung und –lehre ist aber, dass Forschung und Lehre im Begriff Wissenschaft zusammengefasst werden (Wagner, 2000, S. 27). Laut einem Urteil des Bundesverfassungsgerichtes ist durch den Paragraphen jede Tätigkeit geschützt, die „…nach Inhalt und Form als ernsthafter planmäßiger Versuch zur Ermittlung der Wahrheit anzusehen ist“ [BVerfGE 35,79 vom 29.05.1973]. Dabei ist vor allem wichtig, dass das Ergebnis keine Relevanz darauf hat, ob die vorhergehende Wissenschaft schützenswert ist. \textcolor{red}{TODO Schranken}

\section*{Nico}
TODO
\section*{Brian}
TODO
\section*{Kristina}
TODO

\end{document}