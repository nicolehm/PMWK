\documentclass{pmwk}
\usepackage{ngerman}
\usepackage[utf8]{inputenc}
\usepackage{color}
\usepackage{xcolor}
\usepackage[colorinlistoftodos]{todonotes}
\usepackage{amsmath}
\usepackage{amssymb}
\usepackage{amsthm}
\usepackage{listings}
\usepackage{hyperref}
\usepackage{breakurl}
\usepackage{algorithm2e}
\begin{document}

\Abgabeblatt{}{}{}{}




\section*{Glaucia}
TODO
\section*{Elena}
TODO
\section*{Dana}
Ein Staat hat besonderes Interesse daran die Wissenschaft zu schützen und zu fördern. Einerseits um politische Entscheidungen gegenüber der Gesellschaft zu begründen (Huber, 2008, S. 11), andererseits bringt eine florierende Wissenschaft einen wirtschaftlichen Fortschritt des Staates (Huber, 2008, S. 35). Für die Wissenschaft schafft der Staat im Gegenzug eine Rechtsordnung in der sie sich entfalten kann und finanziert sie. Grundlage dafür ist in Deutschland Paragraph 3 Artikel 5 Meinungsfreiheit des Grundgesetzes:\par
„Kunst und Wissenschaft, Forschung und Lehre sind frei. Die Freiheit der Lehre entbindet nicht von der Treue zur Verfassung.“ [GG Art 5 § 3]\par
Es gibt verschiedene herrschende Ansichten über die Auslegung des Paragraphen. Gängige Meinung der Rechtsprechung und –lehre ist aber, dass Forschung und Lehre im Begriff Wissenschaft zusammengefasst werden (Wagner, 2000, S. 27). Laut einem Urteil des Bundesverfassungsgerichtes ist durch den Paragraphen jede Tätigkeit geschützt, die „…nach Inhalt und Form als ernsthafter planmäßiger Versuch zur Ermittlung der Wahrheit anzusehen ist“ [BVerfGE 35,79 vom 29.05.1973]. Dabei ist vor allem wichtig, dass das Ergebnis keine Relevanz darauf hat, ob die vorhergehende Wissenschaft schützenswert ist. \textcolor{red}{TODO Schranken}

\section*{Nico}
TODO
\section*{Brian}
%TODO proofread and spellcheck
Wie finanzieren die Hochschulen ihre Forschung und wie wird dadurch deren Forschungsfreiheit beeinflusst? Zum einen gibt es die Erstmittelfinanzierung. Diese ist aber für unser Referat Relativ uninterressant. Die Erstmittel werden vom zuständigen Ministerium bereitgestellt. Die Drittmittel hingegen sind Ressourcen, welche von aussen gesponsort werden. Dies geschiet entweder privat, kommt aus der Wirtschaft oder wird von Forschungsfördernden Vereinen bereitgestellt. Neben den Vorteilen sind hier besonders die Nachteile interessant. Die Aussenstehenden stellen oft bestimmte Forderungen die erfüllt werden müssen, damit sie Gelder zahlen. Dies sollte man kritisch hinterfragen. Ist Forschung wirklich noch frei, wenn sie durch Aussenstehende eingeschränkt wird? Das DFG (Deutsche Forschungsgemeinschaft) bewertet zudem noch die Forscher anhand der Menge der Drittmittel die sie werben. Hierbei läuft die Gefahr, dass Forscher nicht mehr Forschen sondern, in diesem Falle bewusst übertrieben Formuliert, Geldeintreiber sind. Desweiteren hat sich das Referat mit den Unterschieden der Forschung zwischen Unternehmen und Hochschulen beschäftigt. Wir sind dabei zu dem Schluss gekommen, dass Unternehmen sehr gewinnorientiert Forschen. Die Forschung findet zudem meist hinter geschlossenen Türen statt, so dass die Algemeinheit keinen direkten Zugriff darauf hat. Ausserdem kann es vorkommen, das unerwünschte Forschungsergebnisse einfach vernichtet werden.

\section*{Kristina}
Der Abwurf der Atombomben auf Hiroshima und Nagasaki im Jahr 1945 forderte circa 200000 Menschen und befähigte die Menschheit zudem sich selber zu vernichten. Diese Ausmaße der aus wissenschaftlichen Erkenntnissen entspringenden technischen Macht bewirkte in der Gesellschaft erstmals das Bedürfnis über die Wissenschaft, ihre Methoden und Ziele zu reflektieren. Noch im selben Jahr wurde in den USA die \textit{Society for Social Responsibility in Science (SSRS)} gegründet. Fortan setzen sich ihre Mitglieder für den verantwortungsvollen Umgang mit Wissenschaft ein und forderten den Dialog zwischen Vertretern der Wissenschaft und Gesellschaft. Die SSRS war Vorbild für eine Zahlweltweit entstehender Gesellschaften unter anderem für die \textit{Gesellschaft für Verantwortung in der Wissenschaft (GVW)} im deutschen Sprachraum. Neben den Vereinigungen, die für alle Wissensgebiete offen sind, haben sich eine Zahl Gesellschaften gegründet, die sich für die Verantwortung in einem Spezialgebiet einsetzen. Das \textit{Forum für InformatikerInnen für Frieden und gesellschaftliche Verantwortung e.V. (FIfF)} verfolgt unter anderem das Ziel die Öffentlichkeit über Entwicklungen aus der Informatik und Informationstechnik aufzuklären.
Unabhängig von einer Mitgliedschaft in einer Gesellschaft kann ein Beitrag zur Verantwortung geleistet werden, in dem sich der Forscher der Verantwortung annimmt „…mit allen für ihn exsitierenden Möglichkeiten nach bestem Wissen und Gewissen darauf einzuwirken, dass die Verwendung seiner Produkte in der Gesellschaft in einer für ihn und andere als positiv zu bezeichnenden Art und Weise erfolgt“. Nach Axel Diefenbach kann dies zum Beispiel passieren indem der Forscher neben der Wahrung einer guten wissenschaftlichen Praxis das Engagement in der Gesellschaft zur Aufklärung und Förderung eines Meinungsbildes fördert sowie Normen und Werte der jeweiligen Gesellschaft einhält.

\end{document}