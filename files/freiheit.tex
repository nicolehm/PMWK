\begin{frame}
\frametitle{Freiheit der Wissenschaft an der Universität}
TODO
\end{frame}

\subsection*{Zivilklausel}
\begin{frame}
\frametitle{Zivilklausel}
\begin{itemize}
\item Selbstverpflichtung von wissenschaftlichen Einrichtungen ausschließlich für zivile Zwecke zu forschen. \cite{ZivKlausel}
\item Die erste Zivilklausel trat 1986 an der Universität Bremen in Kraft.
\end{itemize}

\end{frame}

\begin{frame}
\frametitle{Liste mit Hochschulen mit einer Zivilklausel}
\begin{multicols}{2}
\begin{itemize}
\item TU Berlin
\item Uni Bremen
\item Uni Konstanz
\item TU Dortmund
\item Uni Oldenburg
\item TU Ilmenau
\item Uni Tübingen
\item Uni Rostock
\item HS Bremen
\item HS Bremerhaven
\item TU Darmstadt
\item Uni Göttingen
\item Uni Frankfurt am Main
\item Uni Münster
\end{itemize}
\end{multicols}

\end{frame}

\begin{frame}
\frametitle{Pro und Contra Zivilklausel}
\begin{table}
\begin{tabularx}{\linewidth}{>{\parskip1ex}X@{\kern4\tabcolsep}>{\parskip1ex}X}
\toprule
\hfil\bfseries Pros
&
\hfil\bfseries Cons
\\\cmidrule(r{3\tabcolsep}){1-1}\cmidrule(l{-\tabcolsep}){2-2}

%% PROS, seperated by empty line or \par
Der Zwei-plus-Vier-Vertrag besagt, \glqq dass von deutschem Boden nur Frieden ausgehen wird. \grqq \par
Universitäten könnten Teil der Kriegsmaschinerie werden. Bzw. sind es schon.\par
Auch ohne Zivilklausel ist die Wissenschaft nicht frei und wird von den Interessen der Industrie gelenkt.\par

&

%% CONS, seperated by empty line or \par
Einige Klauseln verstoßen gegen die im Grundgesetz garantierte Freiheit von Forschung und Lehre \cite{JKrause} \par

\\\bottomrule
\end{tabularx}
\caption{Pro und Contra Zivilklausel}
\end{table}

\end{frame}
