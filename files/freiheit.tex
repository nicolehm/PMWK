\begin{frame}
\frametitle{Freiheit der Wissenschaft an der Universität}
selbst einschränkungen in der Wissenschaft
    Zivilklausel
    ethikkommissionen? Rechtlich?
selbst einschränkungen in der Lehre
    Akkriditionsverfahren
\end{frame}

\subsection*{Zivilklausel}
\begin{frame}
\frametitle{Zivilklausel}
\begin{itemize}
\item Selbstverpflichtung von wissenschaftlichen Einrichtungen ausschließlich für zivile Zwecke zu forschen. \cite{ZivKlausel}
\item Die erste Zivilklausel trat 1986 an der Universität Bremen in Kraft.
\end{itemize}

\end{frame}

\begin{frame}
\frametitle{Liste mit Hochschulen mit einer Zivilklausel}
\begin{multicols}{2}
\begin{itemize}
\item TU Berlin
\item Uni Bremen
\item Uni Konstanz
\item TU Dortmund
\item Uni Oldenburg
\item TU Ilmenau
\item Uni Tübingen
\item Uni Rostock
\item HS Bremen
\item HS Bremerhaven
\item TU Darmstadt
\item Uni Göttingen
\item Uni Frankfurt am Main
\item Uni Münster
\end{itemize}
\end{multicols}

\end{frame}

\begin{frame}
\frametitle{Pro und Contra Zivilklausel}
\tiny{
\begin{table}
\begin{tabularx}{\linewidth}{>{\parskip1ex}X@{\kern4\tabcolsep}>{\parskip1ex}X}
\toprule
\hfil\bfseries Pros
&
\hfil\bfseries Cons
\\\cmidrule(r{3\tabcolsep}){1-1}\cmidrule(l{-\tabcolsep}){2-2}

%% PROS, seperated by empty line or \par
Der Zwei-plus-Vier-Vertrag besagt, \glqq dass von deutschem Boden nur Frieden ausgehen wird. \grqq \par
Universitäten könnten Teil der Kriegsmaschinerie werden. Bzw. sind es schon.\par
Auch ohne Zivilklausel ist die Wissenschaft nicht frei und wird von den Interessen der Industrie gelenkt.\par
Wissenschaft hat das Potenzial, großen Schaden anzurichten. (Atombombe) \par
Die Zivilausel ist \glqq ein Merkmal der Universität, das man auch nach außen tragen kann.\grqq \cite{ohbtaz}\par
\dots \par

&

%% CONS, seperated by empty line or \par
Einige Klauseln verstoßen gegen die im Grundgesetz garantierte Freiheit von Forschung und Lehre. \cite{JKrause} \par
Eine Zivilklausel ist Realitätsfern. (Dual use ist nicht zu vermeiden) \par
Militärforschung hat immer wieder großen Nutzen für die Zivilnutzung erbracht. \par
Eine Zivilklausel schreckt mögliche private Investoren ab. \par
\dots \par
\\\bottomrule
\end{tabularx}
\caption{Pro und Contra Zivilklausel}
\end{table}
}
\end{frame}

\subsection*{Situation an der Universität Bremen}
\begin{frame}
\includegraphics[scale=0.5]{images/zivilklausel.jpg}
Quelle: http://jetzt.sueddeutsche.de/upl/images/user/qu/quentin-lichtblau/text/regular/901608.jpg
2011 wurde die Zivilklausel der Universität Bremen erfolgreich verteidigt.
\end{frame}

\subsection*{Akkriditionsverfahren}
\begin{frame}
Akkriditionsverfahren dienen der Qualitätssicherung von Studiengängen. Sie sind selbst
\end{frame}

\begin{frame}
Kritik:
\glqq Die Akkreditierung in Deutschland ist teuer, bürokratisch, langsam, ineffizient, rechtlich zweifelhaft und autonomiefeindlich.\grqq Professor Dr. Bernhard Kempen Präsident des DHV
\end{frame}
